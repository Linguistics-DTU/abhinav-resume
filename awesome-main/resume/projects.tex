\cvsection{Projects}
\textbf{Both of these projects have been open sourced and all the assets, including the methodology has been uploaded to Github under the organization Lingistics-DTU \url{https://github.com/Linguistics-DTU}}
\newline
\begin{cventries}
  \cventry
    {People's reception of the banning spree in India}
    {Major Project - 1, 7th Semester }
    {New Delhi, India }
    {November 2015}
    {
      \begin{cvitems}
        \item {We explored the correlations between the various Census data about a particular state\\ \hspace{1cm} and the general "acceptance" of a particular ban. }
        \item {The analysis was done using the Twitter API and the Indian Census data of 2011 }
        \item {Gained extensive knowledge of various Python libraries like, }
\begin{itemize}
  \item For natural language processing - Natural Language ToolKit (NLTK)
  \item For visualization and big-data analysis - Pandas, NumPy, Bokeh
\end{itemize}        
        \item{\textbf{The Result} of the entire analysis were a couple \textbf{iPython notebooks} with interactive visualizations}
        \item{The source code and the project report PDF can be found at \\ \url{https://github.com/Linguistics-DTU/DTU_7th_Sem_Project}}
      \end{cvitems}
    }

 \cventry
    {Language, Space and Mind}
    {Major Project - 2, 8th Semester }
    {New Delhi, India }
    {November 2015}
    {
      \begin{cvitems}
        \item {Explore the usage of Category Theory ( Haskell oriented ) to various Linguistics Models }
        \item {We based this project on the exploratory grounds based on the book by \textbf{Paul Chilton (2014) - Language, Space and Mind}}    
		\item {Gained knowledge about modeling language parser in a strictly Functional programming environment such as Haskell} 
		\item {Explored various approaches towards a more mathematically rigorous semantic/syntactic analysis approach to computational linguistics}         
        \item{\textbf{The Result} of the entire analysis were a couple \textbf{iPython notebooks} with the \textbf{NLTK} analysis of about 20 languages and the current state of quality of parsers}
    	\item{Using this analysis we outline the major steps we need to take to create more Mathematically oriented parsers for syntactic analysis and modeling languages}    
        \item{The source code and the project report PDF can be found at \\ \url{https://github.com/Linguistics-DTU/DTU_8th_Sem_Project}}
      \end{cvitems}
    }


\end{cventries}
